\documentclass[./main.tex]{subfiles}
\begin{document}

\begin{rmk}
  In this section we fulfil the promise of
  computing the homotopy category of a $(\EE , \s{W})$
  given a model structure $(\s{C} , \s{F})$.
  We follow the exposition of Joyal's nCatLab.
  Throughout this section, 
  we fix a model catgory $(\EE , \s{W} , \s{C} , \s{F})$. 

  We have already seen $\HO\,\EE \simeq \HO\,\EE_{fc}$.
  So the goal is to give a nice description of morphisms in $\HO\,\EE_{fc}$.
\end{rmk}
  
\begin{dfn}[Cylinder Object, Left Homotopy , Path Object , Right Homotopy]
  
  Let $A \in \EE$.
  A \emph{cylinder object for $A$} is a factoring 
  \footnote{
    We don't require $IA \to A$ to be a fibration
    because in computations it is useful not to.
    For example, in simplicial sets $X \times \De^1 \to X$
    is in general not a fibration.
  }
  \begin{cd}
    A + A \ar[r,"\in \s{C}"] \ar[rd,"{\id{} + \id{}}"{swap}]
      & IA \ar[d,"\in \s{W}"]\\
      & A
  \end{cd}
  (not necessarily unique). 

  Let $f, g : A \to X$ in $\EE$. 
  Then a \emph{left homotopy}, $h : f \map{l}{} g$, 
  is defined as a factoring with respect to a cylinder object $IA$ of $A$ : 
  \begin{cd}
    A + A \ar[rd,"{f + g}"{swap}] \ar[r] 
      & IA \ar[d,"h"] \\
      & X
  \end{cd}

  The above definition dualises.
  Let $X \in \EE$.
  A \emph{path object for $X$} is a factoring 
  \begin{cd}
    {X} & \\
    {P X} & X \times X \\
    \arrow[from=1-1, to=2-2, "{(\id{} , \id{})}"]
    \arrow[from=1-1, to=2-1, "{\in \s{W}}"']
    \arrow[from=2-1, to=2-2, "{\in \s{F}}"']
  \end{cd}
  Let $f , g : A \to X$ in $\EE$.
  Then a \emph{right homotopy} $h : f \map{r}{} g$ is defined as a a factoring
  of $(f , g)$ with respect to a path object $P X$ of $X$ : 
  \begin{cd}
    A \\
    PX & {X\times X}
    \arrow["h"', from=1-1, to=2-1]
    \arrow[from=2-1, to=2-2]
    \arrow["{(f , g)}", from=1-1, to=2-2]
  \end{cd}
  
\end{dfn}

\begin{prop}[The Homotopy Category of a Model Category]
  
  Let $A , X \in \EE$.
  Then we have the following : 
  \begin{enumerate}
    \item Given $A$ is cofibrant, 
    the left homotopy relation on $\EE(A,X)$ is an equivalence relation.
    Letting $\pi^l (A , X)$ denote the quotient by left homotopy equivalence,
    we have a functor $\pi^l(A , \_) : \EE \to \SET$.
    
    Dually,
    given $X$ is fibrant, 
    the right homotopy relation on $\EE(A,X)$ is an equivalence relation.
    Letting $\pi^r (A , X)$ denote the quotient by left homotopy equivalence,
    we have a functor $\pi^r(\_, X) : \EE\op \to \SET$.

    \item Given $A$ is cofibrant, 
    the left homotopy relation on $\EE(A,X)$ implies 
    the right homotopy relation.

    Dually, given $X$ is fibrant,
    the right homotopy relation on $\EE(A,X)$ implies 
    the left homotopy relation.

    Hence for $A$ cofibrant and $X$ fibrant,
    the two homotopy relations coincide.
    Letting $\pi (A , X) := \pi^r (A , X) = \pi^l (A , X)$,
    we obtain a functor $\pi : \EE_c\op \times \EE_f \to \SET$.

    \item Let $\pi \EE_{fc}$ denote the category defined by the data : 
    \begin{itemize}
      \item objects are fibrant-cofibrant objects of $\EE$
      \item The functor $\pi : \EE_c\op \times \EE_f \to \SET$
      restricts to a functor $\pi : \EE_{fc}\op \times \EE_{fc} \to \SET$.
      We use this for the morphism functor.
    \end{itemize}
    There is an obvious functor $\EE_{fc} \to \pi \EE_{fc}$.
    The result is that this exhibits $\pi \EE_{fc}$ as
    the localisation of $\EE_{fc}$ at weak equivalences.
    Furthermore,
    a morphism $f : A \to X$ in $\EE_{fc}$ is a weak equivalence
    iff it is an isomorphism in $\pi \EE_{fc}$.

    \item Given the following square consists of equivalences 
    \begin{cd}
      {\mathbf{Ho}\,\mathcal{E}_{fc}} & {\mathbf{Ho}\,\mathcal{E}_f} \\
      {\mathbf{Ho}\,\mathcal{E}_c} & {\mathbf{Ho} \, \mathcal{E}}
      \arrow["\simeq"', from=1-1, to=2-1]
      \arrow["\simeq"', from=2-1, to=2-2]
      \arrow["\simeq", from=1-1, to=1-2]
      \arrow["\simeq", from=1-2, to=2-2]
    \end{cd}
    we have that a morphism $f$ in $\EE$ is acyclic
    iff it is inverted in $\HO\,\EE$.
  \end{enumerate}
\end{prop}
\begin{proof}
  
  \textit{(1)}
  Let $A$ be cofibrant. 
  The left homotopy relation is already reflexive and symmetric.
  To show transitivity,
  suppose we have two left homotopies : 
  \begin{cd}
    {A + A} & IA & {A + A} & JA \\
    & X && X
    \arrow["i", tail, from=1-1, to=1-2]
    \arrow["\sim", "{H}"', from=1-2, to=2-2]
    \arrow["{f , g}"', from=1-1, to=2-2]
    \arrow["j", tail, from=1-3, to=1-4]
    \arrow["{g,h}"', from=1-3, to=2-4]
    \arrow["\sim", "{H_1}"', from=1-4, to=2-4]
  \end{cd}
  The key is that we can glue the two cylinders $IA$ and $JA$ together
  by the ``end side of $IA$'' and the ``start side of $JA$''
  to form another cylinder $KA$ whose start is $f$ and end is $h$.
  More concretely, we define $KA$ by the following pushout : 
  \begin{cd}
    A & JA \\
    IA & KA
    \arrow["{i_1}"', from=1-1, to=2-1]
    \arrow[from=2-1, to=2-2]
    \arrow["{j_0}", from=1-1, to=1-2]
    \arrow[from=1-2, to=2-2]
    \arrow["\lrcorner"{anchor=center, pos=0.125, rotate=180}, 
      draw=none, from=2-2, to=1-1]
  \end{cd}
  Define $k : A + A \to KA$ by setting $k_0 := A \to IA \to KA$ via $i_0$
  and $k_1 := A \to JA \to KA$ via $j_1$.
  Then the left homotopies $H, H_1$ glue to give
  a morphism $H_2 : KA \to X$ such that $H_2 k_0 = f$ and $H_2 k_1 = h$.
  It remains to show that $KA$ is a cylinder object for $A$
  via $k : A + A \to KA$ and $KA \to A$.

  First, let's show $KA \to A$ is in $\s{W}$.
  This morphism is defined by the diagram : 
  \begin{cd}
    A & JA \\
    IA & KA \\
    & & A
    \arrow["{i_1}"', from=1-1, to=2-1]
    \arrow[from=2-1, to=2-2]
    \arrow["{j_0}", from=1-1, to=1-2]
    \arrow[from=1-2, to=2-2]
    \arrow["\lrcorner"{anchor=center, pos=0.125, rotate=180}, 
      draw=none, from=2-2, to=1-1]
    \arrow["\sim", from=1-2, to=3-3, bend left = 15]
    \arrow["\sim"', from=2-1, to=3-3, bend right = 15]
    \arrow[from=2-2, to=3-3]
  \end{cd}
  By $\s{W}$ having two-out-of-three,
  it suffices to show that $JA \to KA$ is in $\s{W}$.
  Since $\s{W} \cap \s{C}$ is closed under cobase change,
  it suffices to show that the side inclusion $i_0 : A \to IA$ is
  an acyclic cofibration.
  \begin{lem}
    
    Given a cylinder object $(IA , i , \si)$ of a cofibrant $A$,
    we have $i_0 : A \to IA$ in $\s{W} \cap \s{C}$.

    There is a dual result for path objects of fibrant objects.
    \begin{proof1}
      It is acyclic by two-out-of-three for $\s{W}$.
      And it is a cofibration because it is the composition
      $A \to A + A \to IA$ where 
      $i : A + A \to IA$ is in $\s{C}$ by definition of cylinder objects
      and $A \to A + A$ is the cobase change of $\nothing \to A$,
      a cofibration as assumed.
    \end{proof1}
  \end{lem}
  
  Next, we show $k : A + A \to KA$ is a cofibration.
  This morphism fits into the following commutative diagram
  \begin{cd}
    & A & \varnothing \\
    {A + A + A + A} & {A + A + A} & {A + A} \\
    {IA + JA} & KA
    \arrow["{i , j}"', tail, from=2-1, to=3-1]
    \arrow[from=3-1, to=3-2]
    \arrow["{\id{} , (\id{} , \id{}) , \id{}}", from=2-1, to=2-2]
    \arrow["{i_0 , \_ , j_1}"', tail, from=2-2, to=3-2]
    \arrow["\lrcorner"{anchor=center, pos=0.125, rotate=180}, 
      draw=none, from=3-2, to=2-1]
    \arrow[tail, from=2-3, to=2-2]
    \arrow[tail, from=1-3, to=1-2]
    \arrow[from=1-2, to=2-2]
    \arrow[from=1-3, to=2-3]
    \arrow["\lrcorner"{anchor=center, pos=0.125, rotate=90}, 
      draw=none, from=2-2, to=1-3]
    \arrow["k", tail, from=2-3, to=3-2]
  \end{cd}

  \textit{(2)}
  It suffices to prove the following stronger fact : 
  \begin{lem}

    Let $(IA , i , \si)$ be a cylinder object for $A$ and
    $(PX , \rho , e)$ a path object for $X$.
    Suppose we have an ``open box of homotopies''
    \begin{cd}
      {f_{00}} & {f_{01}} \\
      {f_{10}} & {f_{11}}
      \arrow["{h_0}", from=1-1, to=1-2]
      \arrow["{h_1}"', from=2-1, to=2-2]
      \arrow["{v_0}"', from=1-1, to=2-1]
    \end{cd}
    where $f_{jk} : A \to X$,
    $v_0 : f_{00} \map{}{r} f_{10}$ and 
    $h_j : f_{j 0} \map{}{l} f_{j 1}$.
    Suppose $A$ is cofibrant.
    Then there exists $H : IA \to PX$ such that
    $e_k H i_j = f_{j k}$.
    Pictorially, we can ``complete the box with a \emph{double homotopy}''.
    \begin{cd}
      {f_{00}} & {f_{01}} \\
      {f_{10}} & {f_{11}}
      \arrow["{h_0}", from=1-1, to=1-2]
      \arrow["{h_1}"', from=2-1, to=2-2]
      \arrow["{v_0}"', from=1-1, to=2-1]
      \arrow["{v_1}", from=1-2, to=2-2]
      \arrow["{H}", from=1-1, to=2-2]
    \end{cd}
    
    \begin{proof1}
      The lemma can be formulated as finding a solution to
      the following lifting problem
      \begin{cd}
        A & PX \\
        IA & {X \times X}
        \arrow["{i_0}"', "\sim", tail, from=1-1, to=2-1]
        \arrow[two heads, from=1-2, to=2-2]
        \arrow["{h_0 , h_1}"', from=2-1, to=2-2]
        \arrow["{v_0}", from=1-1, to=1-2]
      \end{cd}
      But we know that given $A$ cofibrant,
      we have $i_0 \in \s{W} \cap \s{C}$.
      Hence we have a solution since fibrations
      right lift against acyclic cofibrations.
    \end{proof1}
  \end{lem}
  Now, given $A$ cofibrant and a left homotopy $h_1 : f \map{}{l} g$,
  we get a right homotopy $v_1 : f \map{}{r} g$ by
  completing the following open box of homotopies
  \begin{cd}
    f & f \\
    f & g
    \arrow["{h_1}"', from=2-1, to=2-2]
    \arrow[from=1-1, to=2-1]
    \arrow[from=1-1, to=1-2]
    \arrow["{v_1}", dashed, from=1-2, to=2-2]
    \arrow["H"{description}, dashed, from=1-1, to=2-2]
  \end{cd}
  where the unnamed homotopies are from reflexivity of 
  the two homotopy relations.

  \textit{(3)}
  First, we show that $\EE_{fc} \to \pi\EE_{fc}$ inverts weak equivalences.
  To that end, 
  we prove a lemma which can be used to show
  when a functor from cofibrant objects is homotopical.

  \begin{lem}[Kenneth Brown]
    
    Let $F : \EE_c \to C$ where $C$ is a category with
    a class of morphisms $W$ satisfying identities and two-out-of-three.
    Suppose $F(\s{W} \cap \s{C}) \subs W$.
    Then $F(\s{W}) \subs W$.

    There is a dual result for $F : \EE_f \to C$
    with $F(\s{W} \cap \s{C}) \subs W$.
    \begin{proof1}
    
      The new idea is that of a \emph{mapping cylinder}.
      Given a morphism $f : A \to B$,
      a \emph{mapping cylinder factorisation} of $f$ is
      a factorisation 
      \begin{cd}
        {A + B} \\
        {C(f)} & B
        \arrow["{f , \id{}}", from=1-1, to=2-2]
        \arrow[tail, from=1-1, to=2-1]
        \arrow["\sim"', from=2-1, to=2-2]
      \end{cd}
      Let $f : A \to B$ in $\EE_c$ be acyclic.
      Via the mapping cylinder factorisation above and
      the fact that $A$ is cofibrant, 
      we have 
      \begin{cd}
        A \\
        {C(f)} & B
        \arrow["f", from=1-1, to=2-2]
        \arrow["{i_A}"', tail, from=1-1, to=2-1]
        \arrow["{q_B}"', from=2-1, to=2-2]
      \end{cd}
      For $F(f) \in W$, it suffices that $F(i_A), F(q_B) \in W$.
      We have $F(i_A)$ because it is an acyclic cofibration by
      two-out-of-three with $q_B, f \in \s{W}$.
      For $F(q_B)$, note that since we have 
      \begin{cd}
        B \\
        {C(f)} & B
        \arrow["{\id{}}", from=1-1, to=2-2]
        \arrow["{i_B}"', "{\sim}", tail, from=1-1, to=2-1]
        \arrow["{q_B}"', from=2-1, to=2-2]
      \end{cd}
      it suffices to show that $F(i_B) \in W$
      by two-out-of-three for $W$.
      Indeed we have $F(i_B) \in F(\s{W} \cap \s{C}) \subs W$ by assumption.

      That technically completes the proof, 
      but let us make a remark on how to make mapping cylinders
      for intuition's sake.
      Such a factorisation can be obtained from 
      a cylinder object $(IA , i , \si)$ of $A$ when it is cofibrant 
      by the pushout 
      \begin{cd}
        {A + A} & IA \\
        {A + B} & {C(f)} \\
        && B
        \arrow["i", tail, from=1-1, to=1-2]
        \arrow["{\id{} , f}"', from=1-1, to=2-1]
        \arrow[from=1-2, to=2-2]
        \arrow[tail, from=2-1, to=2-2]
        \arrow["\lrcorner"{anchor=center, pos=0.125, rotate=180}, 
          draw=none, from=2-2, to=1-1]
        \arrow["{f , \id{}}"{description}, from=2-1, to=3-3, bend right = 15]
        \arrow["{f \si}", from=1-2, to=3-3, bend left = 15]
        \arrow["{q_B}"{description}, dashed, from=2-2, to=3-3]
      \end{cd}
      To show $q_B \in \s{W}$ , it suffices to show
      that the composition $i_B : B \to A + B \to C(f)$ is acyclic.
      This fits into the commutative diagram of pushout squares 
      \begin{cd}
        A & {A + A} & IA \\
        B & {A + B} & {C(f)}
        \arrow["i", tail, from=1-2, to=1-3]
        \arrow["{\id{} , f}", from=1-2, to=2-2]
        \arrow[from=1-3, to=2-3]
        \arrow[tail, from=2-2, to=2-3]
        \arrow["\lrcorner"{anchor=center, pos=0.125, rotate=180}, 
          draw=none, from=2-3, to=1-2]
        \arrow["f"', from=1-1, to=2-1]
        \arrow[from=1-1, to=1-2]
        \arrow[from=2-1, to=2-2]
        \arrow["\lrcorner"{anchor=center, pos=0.125, rotate=180}, 
          draw=none, from=2-2, to=1-1]
      \end{cd}
      So we see that $B \to C(f)$ is a cobase change of 
      $i_0 : A \to IA$.
      Since $A$ is cofibrant, $i_0$ is an acyclic cofibration
      and hence so is $B \to C(f)$.
    \end{proof1}
  \end{lem}
  We will use Yoneda's lemma to show $\EE_{fc} \to \pi\EE_{fc}$ inverts
  weak equivalences.
  Specifically, we show the following : 
  \begin{lem}

    Let $A$ be cofibrant.
    Then $\pi^l(A , \_) : \EE_f \to \SET$ inverts weak equivalences.
    In particular, for an acyclic fibration $f : X \to S$
    where $S$ is cofibrant, the set of sections of $f$
    is singleton up to left homotopy.

    \begin{proof1}
      
      By the lemma of Kennth Brown,
      it suffices to show that $\pi^l(A , \_)$ inverts
      acyclic fibrations.
      So let $f : X \to Y$ be an acyclic fibration.
      Then surjectivity of $\pi^l(A, f)$ is equivalent to 
      solving lifting problems of the form : 
      \begin{cd}
        \varnothing & X \\
        A & Y
        \arrow[tail, from=1-1, to=2-1]
        \arrow[from=2-1, to=2-2]
        \arrow[from=1-1, to=1-2]
        \arrow["\sim", two heads, from=1-2, to=2-2]
        \arrow[dashed, from=2-1, to=1-2]
      \end{cd}
      which we have due to $A$ being cofibrant and 
      the WFS $(\s{C} , \s{W} \cap \s{F})$.
      On the other hand injectivity of $\pi^l(A,f)$ 
      is equivalent to solving lifting problems of the form : 
      \begin{cd}
        {A + A} & X \\
        IA & Y
        \arrow["i"', tail, from=1-1, to=2-1]
        \arrow[from=2-1, to=2-2]
        \arrow[from=1-1, to=1-2]
        \arrow["\sim", two heads, from=1-2, to=2-2]
        \arrow[dashed, from=2-1, to=1-2]
      \end{cd}
      which we also have by the same reason.
    \end{proof1}
  \end{lem}
  We have thus proved that $\EE_{fc} \to \pi \EE_{fc}$ inverts
  weak equivalences.
  So by the UP of localisation,
  we have a commuting triangle : 
  \begin{cd}
    {\mathcal{E}_{fc}} \\
    {\mathbf{Ho}\,\mathcal{E}_{fc}} & {\pi\mathcal{E}_{fc}}
    \arrow[from=1-1, to=2-1]
    \arrow[from=1-1, to=2-2]
    \arrow["T"', from=2-1, to=2-2]
  \end{cd}
  The idea here is that homotopic maps are identified under
  functors which invert weak equivalences.
  \begin{lem}

    Let $F : \EE \to C$ be a functor.
    \begin{enumerate}
      \item If $F$ inverts weak equivalences, then 
      $F$ identifies left homotopic morphisms.
      \item This is also true for functors $F : \EE_{c} \to C$
      \item It is furthermore also true for $F : \EE_{fc} \to C$.
    \end{enumerate}
    \begin{proof1}
      
      \textit{(1)}
      The key is that for any cylinder object $IA$ of an object $A$,
      ``the cylinder collapses''.
      More precisely, the morphism $IA \to A$ gets inverted under $F$,
      and so the two inclusions $i_0, i_1 : A \to IA$ get identified.

      \textit{(2)}
      To reuse the above argument,
      we need only to ensure any cylinder object $IA$ of 
      cofibrant $A$ is also cofibrant.
      But this is true since the composition 
      $\nothing \to A \to A + A \to IA$ consists of cofibrations.

      \textit{(3)}
      We can try again to reuse the above argument,
      in which case we need to prove that any cylinder object $IA$
      of a fibrant-cofibrant object $A$ is also fibrant-cofibrant.
      This is false in general. 
      \footnote{
        In simplicial sets, 
        $X + X \to X \times \De^1 \to X$
        is the standard cylinder object to take. \\
        But one can see that for $X = \De^0$, 
        $\De^1 \to \De^0$ is not an acyclic fibration,
        i.e. $\De^1$ is not a Kan complex.
      }
      Instead, we proceed to show that we only need the existence of \emph{one} 
      fibrant-cofibrant cylinder object $IA$ for each $A$,
      which indeed exists by $(\s{C} , \s{W} \cap \s{F})$-factoring 
      $A + A \to A$.
      \begin{lem}

        Let $A, X \in \EE$ where $X$ is fibrant.
        Then the left homotopy relation on $\EE(A , X)$
        can be defined with respect to a single cylinder object of $A$.

        \begin{proof1}
          Let $f , g : A + A \to X$,
          $(IA , i , \si) , (JA , j , \tau)$ cylinder objects of $A$
          and $H : IA \to X$ a left homotopy from $f$ to $g$.
          By the WFS $(\s{W} \cap \s{C} , \s{F})$,
          the fact that $X$ is fibrant says that 
          we can change $IA$ along any acyclic cofibration $IA \to I'A$.
          \begin{cd}
            IA & X \\
            {I'A} & \bullet
            \arrow["\sim"', tail, from=1-1, to=2-1]
            \arrow[from=2-1, to=2-2]
            \arrow["H", from=1-1, to=1-2]
            \arrow[from=1-2, to=2-2]
            \arrow["{H'}"{description}, dashed, from=2-1, to=1-2]
          \end{cd}
          So we factor $\si : IA \to A$ into an acyclic cofibration 
          followed by a fibraton to get the following commutative diagram : 
          \begin{cd}
            {A+ A} & IA & {I'A} \\
            JA & A & A
            \arrow[tail, from=1-1, to=2-1]
            \arrow[tail, from=1-1, to=1-2]
            \arrow["\sim"', from=2-1, to=2-2]
            \arrow["{\id{}}"', from=2-2, to=2-3]
            \arrow["\sim", tail, from=1-2, to=1-3]
            \arrow[two heads, from=1-3, to=2-3]
            \arrow["\sim"', from=1-2, to=2-2]
            \arrow[dashed, from=2-1, to=1-3]
          \end{cd}
          The outter square is a lifting problem
          to which a solution would give a left homotopy $\tilde{H} : JA \to X$
          from $f$ to $g$.
          But acyclicity of $IA \to A$ implies
          $I'A \to A$ is an acyclic fibration by two-out-of-three for $\s{W}$,
          and so we indeed have a solution.
        \end{proof1}
      \end{lem}
    \end{proof1}
  \end{lem}
  The above result means that we have a well-defined functor $U$ making
  the commutative triangle : 
  \begin{cd}
    {\mathcal{E}_{fc}} & {\pi \mathcal{E}_{fc}} \\
    & {\mathbf{Ho}\,\mathcal{E}_{fc}} 
    \arrow[from=1-1, to=1-2]
    \arrow["U"', from=1-2, to=2-2]
    \arrow[from=1-1, to=2-2]
  \end{cd}
  The claim is that $U, T$ are inverses.
  For this, it suffices that both 
  $\EE_{fc} \to \HO\,\EE_{fc}$ and $\EE_{fc} \to \pi \EE_{fc}$ are
  epimorphisms of categories.
  The former we have seen before.
  For the latter, note that
  $\EE_{fc} \to \pi \EE_{fc}$ is full and bijective on objects.
  This concludes $\pi \EE_{fc} \iso \HO\,\EE_{fc}$ as
  categories under $\EE_{fc}$.

  Finally, let us show that a morphism $f$ in $\EE_{fc}$
  is acyclic iff it it inverted in $\pi\EE_{fc}$.
  The forward direction has been proved.
  For the converse,
  we first note that suffices to do the case that $f$ is a fibration :
  factor $f = pu$ where $u \in \s{W} \cap \s{C}$ and $p \in \s{F}$.
  Then we know already that $u$ is inverted in $\pi\EE_{fc}$.
  So by two-out-of-three for $\s{W}$ and isomorphisms (in $\pi\EE_{fc}$),
  we see that $f \in \s{W}$ iff $p \in \s{W}$ and 
  $f$ is inverted in $\pi\EE_{fc}$ iff $p$ is.
  
  So WLOG assume $f : X \to Y$ is a fibration 
  which is inverted in $\pi\EE_{fc}$.
  This means we have a $g : Y \to X$ with 
  $f g \sim \id{Y}$ and $g f \sim \id{X}$.
  Notice that from $X$ cofibrant, acyclicity of morphisms out of $X$
  is preserved along left homotopies.
  Since $\id{X} \in \s{W}$, we have $g f \in \s{W}$.
  Now \emph{if} $g$ was a ``true'' section of $f$,
  i.e. $f g = \id{Y}$,
  then we would have $f$ as a retract of $g f$.
  \begin{cd}
    X & X & X \\
    Y & X & Y
    \arrow["f"', from=1-1, to=2-1]
    \arrow["f", from=1-3, to=2-3]
    \arrow["g"', from=2-1, to=2-2]
    \arrow["{\id{}}", from=1-1, to=1-2]
    \arrow["{\id{}}", from=1-2, to=1-3]
    \arrow["f"', from=2-2, to=2-3]
    \arrow["{g f}"{description}, from=1-2, to=2-2]
  \end{cd}
  And hence we would be able to conclude that $f$ is acyclic
  since $\s{W}$ is closed under retracts.

  To that end, we seek to replace $g$ with some $s$ where $f s = \id{Y}$.
  To ensure that $s f \in \s{W}$, it suffices to keep $g \sim s$
  for then $s f \sim g f \sim \id{X}$.
  We can indeed find such $s : Y \to X$ as $s = H i_1$
  where $H$ is a solution of the lifting problem 
  \begin{cd}
    Y & X \\
    IY & Y
    \arrow["g", from=1-1, to=1-2]
    \arrow["{i_0}"', tail, from=1-1, to=2-1]
    \arrow[from=2-1, to=2-2]
    \arrow["f", two heads, from=1-2, to=2-2]
    \arrow["H"{description}, dashed, from=2-1, to=1-2]
  \end{cd}
  where the bottom morphism exhibits $f g \sim \id{Y}$.

  \textit{(4)}
  Although we used functorial factorisations to obtain
  the square of equivalence of homotopy categories,
  we won't need functoriality of factorisations to show
  the desired result.

  Let $f : X \to Y$ in $\EE$.
  Then by repeated application of factorisations
  into $(\s{W} \cap \s{C} , \s{F})$ and $(\s{C} , \s{W} \cap \s{F})$,
  we obtain the commutative diagram : 
  \begin{cd}
    & X & Y \\
    & RX & RY & \bullet \\
    \varnothing & WX & WY
    \arrow[two heads, from=2-3, to=2-4]
    \arrow["{f_1}", two heads, from=2-2, to=2-3]
    \arrow[tail, from=3-1, to=3-2]
    \arrow["{f_2}", tail, from=3-2, to=3-3]
    \arrow["f", from=1-2, to=1-3]
    \arrow["\sim"', tail, from=1-2, to=2-2]
    \arrow["\sim", tail, from=1-3, to=2-3]
    \arrow["\sim", two heads, from=3-2, to=2-2]
    \arrow["\sim"', two heads, from=3-3, to=2-3]
  \end{cd}
  This is exactly what we would get from functorial factorisations
  so not much surprises here.
  We have $RX , RY$ fibrant and $W X , W Y$ fibrant-cofibrant.
  Now we see that 
  $f \in \s{W}$ iff $f_1 \in \s{W}$ iff $f_2 \in \s{W}$
  iff $f_2$ is inverted in $\HO\,\EE_{fc}$
  iff $f_1$ is inverted in $\HO\,\EE_f$
  iff $f$ is inverted in $\HO\,\EE$
  as desired.

\end{proof}

\end{document}