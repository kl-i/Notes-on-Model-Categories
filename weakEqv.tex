\documentclass[./main.tex]{subfiles}
\begin{document}
  
\begin{dfn}[Homotopical Categories]

  A \emph{homotopical category} consists of the following data : 
  \begin{itemize}
    \item a category $M$
    \item a class of morphisms $W$ of $M$
    \item $W$ contains identity morphisms and satisfies \emph{two-out-of-six}
  \end{itemize}

  Morphisms in $W$ are called \emph{weak equivalences}.
  Sometimes $W$ is implicit and we simply say $M$ is a homotopical category.
\end{dfn}

\begin{eg}
  \begin{enumerate}
    \item 
    Given any category $M$,
    the class of isomorphisms satisfies two-out-of-six.
    \item Let $M = \CAT$ the category of (small) categories.
    Declare $W$ to be the class of categorical-equivalences.
    Then $W$ satisfies two-out-of-six by application of
    the previous example on the ``$\Pi_0$'' of categories
    and maps on hom sets.
    \item Let $A$ be an abelian category and $\mathrm{Ch}\,A$
    the category of (unbounded) complexes in $A$.
    Declare $W$ to be the class of \emph{quasi-isomorphisms},
    i.e. the morphisms which induce isomorphisms on cohomology.
    Then $(\mathrm{Ch}\, A , W)$ is homotopical by the previous example.

    This includes the examples : 
    the category $B\MOD$ of (left) modules over an associative ring $B$,
    the catgeory $\mathrm{QCoh} X$ of quasi-coherent sheaves on a scheme $X$.
    \item Let $M = \TOP$ and declare $W$ to be the class of morphisms
    which induce an isomophism on $\pi_0$ and 
    $\pi_{> 0}$ at all points.
    Then $(\TOP , W)$ is again homotopical by the first example.
  \end{enumerate}
\end{eg}

\begin{lem}
  
  Let $M$ be a category and $W$ a class of morphisms of $M$
  satisfying two-out-of-six.
  Then $W$ satisfies two-out-of-three.

  \begin{proof1}
  \end{proof1}
\end{lem}

\begin{prop}[1-Categorical Localisation]

  Let $(M , W)$ be a homotopical category.
  Define the category $M[W \inv]$ by the following data : 
  \begin{itemize}
    \item objects of $M[W \inv]$ are the same as those of $M$.
    \item $M[W \inv](x,y)$ consists of ``fractions of morphisms'' 
    $(f_0 / g_0) \cdots (f_n / g_n)$ where the $g_i$ are in $W$,
    quotiented out by removing any ``$f / g$'' with $f , g \in W$
    and $\id{} / \id{}$.

    These are also sometimes called \emph{zig-zags}.
    \item Morphisms compose by ``multiplication''.
    Checking associativity is tedious.
  \end{itemize}
  There is a functor $M \to M[W\inv]$
  that is identity on objects and 
  maps each morphism $f$ to the equivalence class of $f / \id{}$.

  For any category $N$,
  define $\CAT_W (M , N)$ to be the full subcategory 
  of functors $M \to N$ \emph{inverting $W$},
  i.e. maps morphisms in $W$ to isomorphisms.
  For any functor $l : M \to L(M)$,
  we say \emph{$l$ exhibits $L(M)$ as a (weak) localisation of $M$ at $W$}
  when $l$ inverts $W$ and or any category $N$ we have
  a categorical equivalence : 
  \[
    \_ \circ l : \CAT(L(M) , N) \map{\sim}{} \CAT_W (M , N)
  \]

  Then $M \to M[W\inv]$ exhibits $M[W\inv]$ as 
  a (weak) localisation of $M$ at $W$.
  Furthermore, $M \to M[W\inv]$ is an epimorphism.

  If $W$ is implicit, we use $\HO M$ to denote $M[W\inv]$.
\end{prop}
\begin{proof}
  Nothing unexpected.
  The morphism $M \to M[W\inv]$ is epi because
  for any commuting triangle 
  \begin{cd}
    M \\
    {M[W^{-1}]} & N
    \arrow[from=1-1, to=2-1]
    \arrow["T", from=1-1, to=2-2]
    \arrow["{\tilde{T}}"', from=2-1, to=2-2]
  \end{cd}
  $\tilde{T}$ sends a zig-zag $(f_0/g_0)\cdots (f_n/g_n)$
  to $T(f_0)T(g_0)\inv \cdots T(f_n)T(g_n)\inv$,
  and hence is determined by $T$.
\end{proof}

\begin{rmk}

  Some motivations for \emph{model structures} via
  issues with the 1-categorical localisation : 
  \begin{itemize}
    \item Using ZFC set theory as foundations,
    a common issue is that when $M$ has a proper class of objects 
    (e.g. $M = \SET , \GRP , \RING$ etc),
    the homs of $M[W\inv]$ are also proper classes.
    \item The homs of $M[\inv]$ are hard to get your hands on.
  \end{itemize}
\end{rmk}

\begin{dfn}[Homotopical Functors]

  Let $(M, W_M)$ and $(N , W_N)$ be homotopical categories.
  Then a functor $F : M \to N$ is called \emph{homotopical} when
  $F W_M \subs W_N$.
\end{dfn}

\begin{eg}

  Let $(M,W)$ be a homotopical category and $F : M \to M$
  together with $\al : \id{} \to F$ with components in $W$.
  Then $F$ is homotopical by $W$ satisfying two-out-of-three.

\end{eg}

\begin{ceg}
  
  A list of functors between homotopical categories
  which we wish were homotopical : 
  \begin{enumerate}
    \item Let $I$ be a (small) category and $(M,W)$ a homotopical category.
    Suppose $M$ has colimits of $I$-diagrams, i.e. 
    we have a functor $\COLIM_I : M^I \to M$ left adjoint to 
    the constant diagram functor $M \to M^I$.
    We can give $M^I$ a class of weak equivalences
    by declaring weak equivalences to be the ones that are
    component-wise in $W$.

    It would be nice if $\COLIM_I : M^I \to M$ is homotopical 
    since that would say ``colimits are invariant under $W$''.
    Unfortunately, this is not true in general.

    Take $M = \TOP$ and $I$ the shape of the diagram for a pushout.
    A counterexample is then given by the following morphism of diagrams : 
    \begin{cd}
      {D^{n+1}} & {S^n} & {D^{n+1}} \\
      \bullet & {S^n} & \bullet
      \arrow[from=2-2, to=2-1]
      \arrow[from=2-2, to=2-3]
      \arrow[from=1-2, to=1-1]
      \arrow[from=1-2, to=1-3]
      \arrow[from=1-1, to=2-1]
      \arrow[from=1-3, to=2-3]
      \arrow["{\mathbb{1}}"{description}, from=1-2, to=2-2]
    \end{cd}
    All vertical morphisms are weak-equivalences in $\TOP$
    but the induced map $S^{n+1} \to \bullet$ is 
    certainly not a weak equivalence.
    \item TODO : show how non-exactness of functors from homological algebra
    like tensor and hom can be expressed as non-homotopicalness of the functors.
  \end{enumerate}
\end{ceg}



\end{document}