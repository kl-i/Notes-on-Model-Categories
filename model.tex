\documentclass[./main.tex]{subfiles}
\begin{document}

\begin{rmk}
  Digestion of the exposition on Joyal's nCatLab.
\end{rmk}
  
\begin{dfn}[Model Categeories, Weak factorisation systems, Lifting Properties]

  Let $\EE$ be a category. 
  For $(L,R)$ a pair of classes of morphisms of $\EE$, 
  $(L,R)$ is called a \emph{weak factorisation system} when 
  the following are true : 
  \begin{itemize}
    \item any $f \in \EE^\to$ factorises as 
    \begin{cd}
      A \ar[dr,"f"{swap}] \ar[r,"l \in L"]
        & C \ar[d,"r \in R"]\\
        & B
    \end{cd}
    (possibly non-uniquely)
    \item $L$ is exactly the morphisms that have \emph{left-lifting-property}
    against all morphisms in $R$.
    \item $R$ is exactly the morphisms that have \emph{right-lifting-property}
    against all morphisms in $L$. 
  \end{itemize}

  Let $(\EE , \s{W})$ be a homotopical category where
  $\EE$ is finitely complete and finitely cocomplete.
  A \emph{model structure} is a pair $\s{C}, \s{F}$
  of classes of families of morphisms in $\EE$ such that : 
  \begin{itemize}
    \item $(\s{C} \cap \s{W}, \s{F})$ is a weak factorisation system for $\EE$.
    \item $(\s{C}, \s{W} \cap \s{F})$ is a weak factorisation system for $\EE$. 
  \end{itemize}
  A \emph{model category} is a finitely complete and finitely cocomplete
  homotopical category equipped with a model structure. 
\end{dfn}

\begin{dfn}[Homotopy Category, Weak Equivalences/Acyclic Morphisms, 
  (Co)Fibrations, (Co)Fibrant Objects]
  
  Let $(\EE , \s{W} , \s{C} , \s{F})$ be a model category. 
  \begin{itemize}
    \item morphisms in $\s{W}$ are called \emph{acyclic}. 
    Sometimes, they are also called \emph{trivial}.
    \item morphisms in $\s{C}$ are called \emph{cofibrations}
    \item morphisms in $\s{F}$ are called \emph{fibrations}
    \item objects $X$ where $X \to 1$ is a fibration are called 
    \emph{fibrant}. 
    \item objects $X$ where $\nothing \to X$ is a cofibration are called 
    \emph{cofibrant}. 
    \item objects that are both fibrant and cofibrant, 
    we call \emph{fibrant-cofibrant}.
  \end{itemize}
\end{dfn}

\begin{rmk}
  Many authors (such as Hovey)
  require the two factorisations to be \emph{functorial}.
  This is usually satisfied in practice 
  (in particular, whenever we have a cofibrantly generated model structure).
\end{rmk}

\begin{prop}

  Let $(\EE, \s{W} , \s{C} , \s{F})$ be a model category
  with functorial factorisations.
  The following square of categories consists of equivalences : 
  \begin{cd}
    {\mathbf{Ho}\,\mathcal{E}_{fc}} & {\mathbf{Ho}\,\mathcal{E}_f} \\
    {\mathbf{Ho}\,\mathcal{E}_c} & {\mathbf{Ho} \, \mathcal{E}}
    \arrow["\simeq"', from=1-1, to=2-1]
    \arrow["\simeq"', from=2-1, to=2-2]
    \arrow["\simeq", from=1-1, to=1-2]
    \arrow["\simeq", from=1-2, to=2-2]
  \end{cd}
\end{prop}
\begin{proof}
  The functorial factorisation $(\s{W} \cap \s{C} , \s{F})$
  gives a right deformation of $\EE$ to $\EE_f$
  and hence a quasi-inverse to $\HO\,\EE_f \to \HO\,\EE$.
  Now the functorial factorisation $(\s{C} , \s{W} \cap \s{F})$
  similarly gives a left deformation of $\EE_{f}$ to $\EE_{fc}$.
  The rest is analogous.
\end{proof}

\begin{prop}[Elementary results]

  Let $(\EE, \s{W} , \s{C} , \s{F})$ be a model category. 
  \begin{enumerate}
    \item $\s{C},\s{C} \cap \s{W}, \s{F}, \s{F} \cap \s{W}$ 
    closed under composition and retracts. 
    \item $\s{F}, \s{F} \cap \s{W}$ closed under base change and product. 
    \item $\s{C}, \s{C} \cap \s{W}$ closed under cobase change and coproduct. 
    \item $\s{C} \cap \s{W} \cap \s{F}$ is precisely the isomorphisms. 
    \item (Tierney) $\s{W}$ closed under retracts. (Slightly non-trivial.)
    In particular, identity morphisms are in $\s{W}$.
  \end{enumerate}
\end{prop}
\begin{proof}
  Besides that last result by Tierney,
  the rest is rather formal.
\end{proof}


\end{document}