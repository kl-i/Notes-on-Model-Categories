\documentclass[./main.tex]{subfiles}
\begin{document}
  
\begin{rmk}

  From the Ken Brown lemma,
  we see that for a functor $F : M \to N$ between model categories,
  it is homotopical on $M_c$ if $F$ maps acyclic cofibrations
  between cofibrant objects to weak equivalences.
  In the presence of a functorial factorisations
  in the model structure of $M$,
  we can then compute the left derived functor of $F$.
  This partially motivates the following definition.
\end{rmk}

\begin{dfn}
  
  Let $F : M \to N$ be a functor between model categories.
  Then $F$ is called \emph{left Quillen}
  when $F(\s{C}) \subs \s{C}$ and $F(\s{W} \cap \s{C}) \subs \s{W} \cap \s{C}$.

  The above dualises to \emph{right Quillen} functors.
\end{dfn}

\begin{prop}[Quillen Adjunctions]
  
  Let $F \dashv G : M \leftrightarrow N$ be an adjunction
  where $F$ is left Quillen and $G$ is right Quillen.
  Assume that $M, N$ have functorial factorisations.
  Then the total left derived functor of $F$ and 
  the total right derived functor of $G$ exists,
  together forming an adjunction at the level of homotopy categories.
  \begin{cd}
    {\mathbf{Ho}\,M} && {\mathbf{Ho}\,N}
    \arrow["{\mathbf{L} F}", shift left=3, from=1-1, to=1-3]
    \arrow["{\mathbf{R}G}", shift left=3, from=1-3, to=1-1]
    \arrow["\bot"{description}, draw=none, from=1-1, to=1-3]
  \end{cd}
\end{prop}
\begin{proof}
  Direct application of 
  derived functors via deformations.
\end{proof}

\end{document}