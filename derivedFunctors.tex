\documentclass[./main.tex]{subfiles}
\begin{document}

\begin{rmk}
  Digestion of the exposition in Riehl's \textit{Categorical Homotopy Theory}.
\end{rmk}
  
\begin{dfn}
  
  Let $M, N$ be homotopical categories and
  $F : M \to N$ a functor.

  Let $\ga : M \to \HO M$ and $\de : N \to \HO N$ be
  the obvious functors.

  Then a \emph{total left derived functor of $F$} is defined as
  a \emph{right} Kan extension of $\de F$ along $\ga$.
  Unpacking the definition, this consists of the following data : 
  \begin{itemize}
    \item a functor $\mathbf{L} F : \HO M \to \HO N$
    \item a natural transformation $\la : \mathbf{L} F \ga \to \de F$ such that
    \[
      \_\, \ga : \MOR (\_ , \mathbf{L} F) \map{\iso}{} \MOR (\_ , \de F)
    \]
  \end{itemize}
  i.e. ``$\mathbf{L} F$ is the left-closest extension of $\de F$ along $\ga$''.
  Using the equivalence $\CAT(\HO M , \HO N) \simeq \CAT_W (M , \HO N)$,
  the above is equivalent to specifying 
  a universal morphism $\la : \mathbf{L} F \to \de F$
  from $\CAT_W (M , \HO \, N)$ to $\de F$.

  Often, it is possible to improve the situation by finding
  a homotopical functor $\bb{L} F : M \to N$ with $\la : \bb{L} F \to F$ 
  such that $\de \bb{L} F$ in $\CAT(M , \HO\, N)$ corresponds to 
  a total left derived functor of $F$.
  \[
    \CAT_W (M , \HO N) ( \_ , \de \bb{L} F) 
    \map{\iso}{} \CAT (M , \HO N) ( \_ , \de F)
  \]
  We call such $(\bb{L} F , \la)$ a \emph{left derived functor} of $F$.

  These definitions dualise to give 
  \emph{total right derived functors} and \emph{right derived functors}.
\end{dfn}

\begin{rmk}
  
  The following result gives a way of constructing derived functors.
\end{rmk}

\begin{prop}[Derived Functors via Deformations]
  
  Let $M, N$ be homotopical categories and $F : M \to N$ a functor 
  (not necessarily homotopical).

  Define a \emph{left deformation} of $F$ to consist of the following data : 
  \begin{itemize}
    \item a functor $Q : M \to M$.
    \item a natural transformation $q : Q \to \id{M}$ with
    components in weak equivalences of $M$.
    \item we require that $F : M_Q \to N$ is homotopical
    where $M_Q$ is the full subcategory of $M$ consisting of 
    objects in the image of $Q$, turned into a homotopical category via
    endowing it with weak equivalences from $M$.
  \end{itemize}

  We have the following : 
  \begin{enumerate}
    \item Let $(Q, q)$ be a left deformation of $F$.
    $(FQ , Fq)$ gives a left derived functor for $F$.
    Furthermore,
    this is in fact 
    an \emph{absolute} right Kan extension of $M \to N \to \HO N$
    along $M \to \HO \, M$.
    \item Suppose $F \dashv G : M \rightleftharpoons N$ is an adjunction,
    such that we have a total left derived functor $(\mathbf{L} F , \la)$ of $F$
    and a total right derived functor $(\mathbf{R} G , \rho)$ of $G$
    where both total derived functors are absolute Kan extensions.
    Then we have an adjunction 
    \begin{cd}
      {\mathbf{Ho}\,M} && {\mathbf{Ho}\,N}
      \arrow["{\mathbf{L} F}", shift left=3, from=1-1, to=1-3]
      \arrow["{\mathbf{R}G}", shift left=3, from=1-3, to=1-1]
      \arrow["\bot"{description}, draw=none, from=1-1, to=1-3]
    \end{cd}
  \end{enumerate}
  that is compatible with localisation in the sense that we have 
  the commuting square : 
  \begin{cd}
    {N(F m , n)} & {M(m , G n)} \\
    {\mathbf{Ho}\,N(F m , n)} & {\mathbf{Ho}\, M(m , G n)} \\
    {\mathbf{Ho}\,N(\mathbf{L} F m , n)} & {\mathbf{Ho}\, M (m , \mathbf{R}G n )}
    \arrow[from=1-1, to=2-1]
    \arrow["{\_ \lambda_m}"', from=2-1, to=3-1]
    \arrow[from=1-2, to=2-2]
    \arrow["{\rho_n \_}", from=2-2, to=3-2]
    \arrow["\cong", from=1-1, to=1-2]
    \arrow["\cong"', from=3-1, to=3-2]
  \end{cd}
  
\end{prop}
\begin{proof}
  
  \textit{(1)}
  
  Let $\de : N \to \HO N$ be the obvious functor.
  We directly show that $(\de F Q , \de F q)$ gives an
  absolute right Kan extension of $\de F$ along $M \to \HO \, M$.

  Let $H : \HO\, N \to E$.
  Let $G \in \CAT_W (M , E)$ where 
  $W$ is the class of weak equivalences of $M$. 
  We need to show a bijection : 
  \[
    G \to H \de F Q \SAMEAS
    G \to H \de F
  \]
  via composing with $H \de F q$.
  Let $\al : G \to H \de F$.
  We give a unique $\al_0 : G \to H \de F Q$ such that 
  $\al = (H \de F q) \al_0 $.
  Suppose we have such an $\al_0$.
  By "restricting" the commuting triangle $\al = (H \de F q ) \al_0$ along
  $q : Q \to \id{M}$,
  we have the following commutative diagram : 
  \begin{cd}
    & {H \delta F Q^2} & {H \delta F Q} \\
    GQ & {H \delta F Q} & {H \delta F} \\
    G
    \arrow["{H \delta F q}", from=2-2, to=2-3]
    \arrow["\alpha"{description}, from=3-1, to=2-3]
    \arrow["{\alpha_0}", from=3-1, to=2-2]
    \arrow["{H \delta F q}", from=1-3, to=2-3]
    \arrow["{\alpha Q}"{description}, from=2-1, to=1-3]
    \arrow[from=1-2, to=2-2]
    \arrow["{H \delta F Q q}", from=1-2, to=1-3]
    \arrow["Gq"', from=2-1, to=3-1]
    \arrow["{\alpha_0 Q}", from=2-1, to=1-2]
  \end{cd}
  Now the point is that since $q$ has components in weak equivalences of $M$,
  we have $Gq$ is an isomorphism since 
  $G$ maps weak equivalences to isomoprhisms,
  and $H \de F Q q$ is also an isomorphism since $Q$ is homotopical
  and $F$ is homotopical when restricted to $M_Q$,
  and $H$ preserves isomorphisms.
  We can thus solve for $\al_0$ uniquely as 
  \[
    \al_0 = (H \de F Q q) (\al_0 Q) (G q)\inv
    = (H \de F Q q) (H \de F Q q)\inv (\al Q) (G q)\inv
    = (\al Q) (G q)\inv 
  \]

  \textit{(2)}
  Purely formal. Indefinitely postponed.

\end{proof}

\begin{rmk}
  
  The above proposition gives another reason for model structures : 
  given a not necessarily homotopical functor $F : M \to N$ 
  between homotopical categories,
  a suitable model structure on $M$ will provide
  deformations for $F$.
\end{rmk}

\end{document}